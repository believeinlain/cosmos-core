{\bfseries Please note that the software and its documentation are work in progress. If you find problems or no documentation for something you\textquotesingle{}re looking for please let us know.}

The C\+O\+S\+M\+O\+S/core is the basic element of the C\+O\+S\+M\+OS project with the essential libraries and programs. The api documentation includes the descriptions of the basic support functions necessary to make C\+O\+S\+M\+O\+S/core work, as well as the most useful agent and support programs. The cosmos core is broadly divided into programs and libraries.

If you haven\textquotesingle{}t downloaded or setup the C\+O\+S\+M\+OS software please go to the \href{https://bitbucket.org/cosmos-project/tutorial/wiki/Home}{\tt C\+O\+S\+M\+OS 101 tutorial}. Also for the latest information on C\+O\+S\+M\+OS refer to the website\+: \href{http://www.cosmos-project.org/}{\tt http\+://www.\+cosmos-\/project.\+org/}. The cosmos-\/core api documentation is available on \href{http://cosmos-project.org/docs/core/current/}{\tt http\+://cosmos-\/project.\+org/docs/core/current/} with instructions on how to compile the code in different architectures, the api structure, some tutorials and the description for most classes and functions. Note\+: remember to update the code regularly (git pull).

Build and install cosmos/core and run two agents that talk to each other. These instructions assume that you are in the {\bfseries cosmos/src/core} folder. Make sure you have all the tools required to build C\+O\+S\+M\+OS including Cmake (please check the required setup from \href{https://bitbucket.org/cosmos-project/tutorial/wiki/Setup}{\tt https\+://bitbucket.\+org/cosmos-\/project/tutorial/wiki/\+Setup}).

Quick-\/start\+:

$\ast$$\ast$1) Using the terminal $\ast$$\ast$

Open a terminal and type


\begin{DoxyCode}
#!shell

cd cosmos/src/core/build
cmake ..
make
\end{DoxyCode}


you can make the code compile faster by using multiple cores. This example shows how to use 8 cores\+: 
\begin{DoxyCode}
#!shell

make -j8
\end{DoxyCode}


{\bfseries 2) Using Qt Creator}


\begin{DoxyItemize}
\item Make sure you are using the latest version of Qt and Start Qt Creator
\item Open the C\+Make\+Lists.\+txt file (File -\/$>$ Open File or Project -\/$>$ Open project \+: cosmos/src/core/\+C\+Make\+Lists.\+txt)
\item Select a build directory (or use the default)
\item Optional step to compile the code faster. Please read note 1 if you\textquotesingle{}re interested. (For Linux check note 2)
\item Build the project \+: ctrl + b
\item Check your \char`\"{}cosmos\char`\"{} folder (ex\+: C\+:/cosmos) to see the files that were installed
\item Select agent\+\_\+001. Click on the \char`\"{}\+Project\char`\"{} button (computer icon on left side of Qt Creator) and select agent\+\_\+001
\item Run agent\+\_\+001. Click on the \char`\"{}\+Run\char`\"{} button (big green icon on left side of Qt Creator) or press \textquotesingle{}ctrl+r\textquotesingle{}
\item Select agent\+\_\+002.
\item Run agent\+\_\+002. If running agent\+\_\+002 stops the run of agent\+\_\+001 see Note 2.
\end{DoxyItemize}

The two C\+O\+S\+M\+OS agents should be talking to each other at this point. 

Alternatively you can use the cosmos-\/core.\+pro file (qmake) or cosmos-\/core.\+qbs (Q\+BS) but these are just experimental build options for the moment.


\begin{DoxyItemize}
\item Open Qt Creator and open the project file \textquotesingle{}cosmos-\/core.\+pro\textquotesingle{}
\item Configure your project (ex\+: on Windows it will be something like \textquotesingle{}Desktop Qt 5.\+5.\+1 Min\+GW 32 bit\textquotesingle{})
\item Optional step to compile the code faster. Please read note 1 if interested. (For Linux check note 2)
\item Build the project by pressing the \char`\"{}\+Build\char`\"{} button (hammer icon on left side of Qt Creator), or press \textquotesingle{}ctrl+b\textquotesingle{}
\end{DoxyItemize}

For more detailed instructions to get started with C\+O\+S\+M\+OS please read the \href{https://bitbucket.org/cosmos-project/core/src/master/docs/}{\tt core/docs/\+R\+E\+A\+D\+M\+E.\+md} inside the documentation folder.

\subsection*{Note 1}

This is only valid for Min\+GW and G\+CC compilers (msvc uses Jom to handle multiple cores). To compile the code faster using all the cores on your machine go to Qt Creator -\/$>$ Projects (icon on left side bar) -\/$>$ Build Steps -\/$>$ Make \+: Details (expand the icon) on arguments add \char`\"{}-\/j4\char`\"{} or whatever number of cores that your computer supports. In some cases you may have to add a space in between \char`\"{}-\/j 4\char`\"{}. In some cases it is also possible to just add \textquotesingle{}-\/j\textquotesingle{} and the compiler will automatically use as many processes it can to compile. This approach works well on Windows but in Linux it seems to freeze the computer because it starts more threads than cores. Use it with caution. If you really need super compilation times then install M\+S\+VC 2013 or above. See the results and make your decision.

Compilation tests from cosmos-\/core.\+pro using a Win7 with Qt 5.\+5.\+1 Min\+GW 32 bit, A\+MD F\+X(tm)-\/8120 Eight Core Processor 3.\+11 G\+Hz, 16 GB Ram, 64 bit OS

\tabulinesep=1mm
\begin{longtabu} spread 0pt [c]{*{4}{|X[-1]}|}
\hline
\rowcolor{\tableheadbgcolor}\PBS\raggedleft \textbf{ Kit }&\PBS\centering \textbf{ Build Step }&\PBS\raggedleft \textbf{ Cores used }&\textbf{ Compile time  }\\\cline{1-4}
\endfirsthead
\hline
\endfoot
\hline
\rowcolor{\tableheadbgcolor}\PBS\raggedleft \textbf{ Kit }&\PBS\centering \textbf{ Build Step }&\PBS\raggedleft \textbf{ Cores used }&\textbf{ Compile time  }\\\cline{1-4}
\endhead
\PBS\raggedleft Desktop Qt 5.\+5.\+1 Min\+GW 32bit &\PBS\centering mingw32-\/make.\+exe (default) &\PBS\raggedleft 1 &3m 30s \\\cline{1-4}
\PBS\raggedleft Desktop Qt 5.\+5.\+1 Min\+GW 32bit &\PBS\centering mingw32-\/make.\+exe -\/j4 &\PBS\raggedleft 4 &1m 16s \\\cline{1-4}
\PBS\raggedleft Desktop Qt 5.\+5.\+1 Min\+GW 32bit &\PBS\centering mingw32-\/make.\+exe -\/j8 &\PBS\raggedleft 8 &1m 4s \\\cline{1-4}
\PBS\raggedleft Desktop Qt 5.\+5.\+1 Min\+GW 32bit &\PBS\centering mingw32-\/make.\+exe -\/j9 &\PBS\raggedleft 8 &1m 4s \\\cline{1-4}
\PBS\raggedleft Desktop Qt 5.\+5.\+1 Min\+GW 32bit &\PBS\centering mingw32-\/make.\+exe -\/j &\PBS\raggedleft 8 &58 s \\\cline{1-4}
\PBS\raggedleft Desktop Qt 5.\+5.\+1 Min\+GW 32bit &\PBS\centering jom.\+exe (custom step) &\PBS\raggedleft 8 &1m 16s \\\cline{1-4}
\PBS\raggedleft Desktop Qt 5.\+5.\+1 M\+S\+V\+C2013 64bit &\PBS\centering jom.\+exe (default) &\PBS\raggedleft 8 &30s \\\cline{1-4}
\end{longtabu}
\subsection*{Note 2}

Qt creator on Linux and Windows has an option to close the programs automatically when running another program. You will need to disable this behavior to run the two agents at the same time. Go to Tools $>$ Options $>$ Build and Run $>$ General. Change �\+Stop applications before building\+:� to None. 